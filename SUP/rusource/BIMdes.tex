\documentclass[11pt]{article}
\usepackage{array,epsfig,epsf,rotating,graphics,colordvi,color,a4}
\newcommand{\Mder}[1]{\frac{{\mathrm D}#1}{{\mathrm D}t} }
\newcommand{\half}{\frac{1}{2}}
\newcommand{\Ai}{\mathrm{Ai}}
\newcommand{\third}{\frac{1}{3}}
\newcommand{\sixth}{\frac{1}{6}}
\newcommand{\beq}{\begin{equation}}
\newcommand{\eeq}{\end{equation}}
\newcommand{\beqn}{\begin{eqnarray}}
\newcommand{\eeqn}{\end{eqnarray}}
\newcommand{\der}[2]{\frac{\partial #1}{\partial #2}}
\newcommand{\oder}[2]{\frac{{\rm d}#1}{{\rm d}#2}}
\newcommand{\numm}[1]{(\ref{#1})}
\newcommand{\e}{\mathrm{e}}
\newcommand{\imunit}{{\mathrm i}}
\newcommand{\OBS}[1]{\marginpar{\scriptsize #1}}  % \OBS{margin comment}
\newcommand{\us}{u^{(s)}}
\newcommand{\vs}{v^{(s)}}
\newcommand{\dt}{\Delta t}
\newcommand{\Mdto}[1]{\frac{{\mathrm D}^2#1}{{\mathrm D}t^2} }

\begin{document}
\title{A note on a BIM model made for runup}
\author{G.~Pedersen}
\date{\today }
\maketitle
\subsubsection{Introduction}
This note is a brief description of a full potential flow model with
particular adaptations to moving shorelines.
 The model is closely related to the
higher order technique of  \cite{Cooker:1990,Dold:1992}, while the formulation differs from the standard references \cite{Grilli:1990,Grilli:1994,Grilli:1997}. However, while these references employ high order polynomials for interpolation along the contour
we use 
cubic splines. This makes inclusion of boundary conditions simpler and does 
 allow for the inclusion of a moving shoreline in particular. 

At the shoreline point we 
assume analyticity which in principle excludes cases with contact angles 
larger than $90^\circ$. Also some other features, such as the numerical 
integration procedure along the contours, differ from the references. 
Boundary integral methods are not efficient for the computation of very thin swash tongues that evolve during runup/withdrawal of higher waves. This is ironical
since the flow may be so simple that even the shallow water equations are unnecessarily complicated (see \cite{Jensen:2003}). The proximity of the surface and 
the bottom parts of the contour requires a smaller time step relative to 
the grid spacing and a higher order numerical integration than in deeper water.
A runup value should then only be accepted if a systematic refinement sequence 
of at least three grids, with the the same integration rule, produce consistent results and point to at least three correct digits.      



\section{A boundary integral method}
In the fluid the motion is governed by the Laplace equation 
\[\nabla^2\phi=0\quad \mathrm{for}\quad -h<z<\eta.\]
At the free surface ($z=\eta$) the Bernoulli equation is
expressed as
\beq
\label{Bernoulli}
\Mder\phi -\half(\nabla\phi)^2+\eta=0
\eeq
The kinematic condition at the surface is written in the Lagrangian form
\[\Mder\eta=\der\phi z\,\quad \Mder\xi=\der\phi x,\]
where $(\eta,\xi)$ is the position of a surface particle.
At rigid boundaries (bottom or sidewalls) we have
\[\der\phi n=0,\] 
where $n$ denotes the direction normal to the boundary.

This model is related to the high order technique of  
\cite{Dold:1992}. However, to allow more flexible boundary conditions, 
as sloping beaches, the high order polynomials are replaced by cubic splines 
for the spatial interpolation between nodes. Accordingly the order of the 
temporal scheme is reduced to third order accuracy. 
 The key features then become
\begin{itemize}
\item Lagrangian particles are used along the free surface. At other 
boundaries both fixed and moving nodes may be employed
\item Cauchy's formula for complex velocity ($q=u-\imunit v$) is used to produce an implicit relation between the velocity components along the surface
\beq
\label{Cauchyrel} \alpha \imunit q(z_p)=\mathrm{PV} \oint\limits_C \frac{q(z)}{z_p-z}dz
\eeq
where $\alpha$ is the interior angle. 
Following \cite{Dold:1992} the integral equation \numm{Cauchyrel} is rephrased
in terms of the velocity components tangential and normal to the contour,
denoted by $\us$ and $\vs$, respectively. Invoking the relation 
\[\us-\imunit \vs =\mathrm{e}^{\imunit\theta}(u-iv),\]
where $\theta$ is the angle between the tangent and the $x$-axis,
we then obtain 
\beq
\label{usrel} \alpha \imunit (\us_p -\vs_p)=\mathrm{e}^{\imunit\theta_p}\mathrm{PV} \oint\limits_C \frac{\us-i\vs}{z_p-z}ds.
\eeq
For rigid boundaries, where the normal velocity is known, the real component
of this equation is imposed, while the imaginary component is used at free surfaces, where $u_s$ is known from the integration of the Bernoulli equation.
The equation set is established by
collocation in the sense that $z_p$ runs through all nodes to produce
as many equations as unknowns (see figure \ref{Cauchydef}). 
\item Cubic splines for field variables -- solution is twice continuously
differentiable
\item Combination of Taylor series expansion and multi-step
technique used for time integration. This allows for variable time stepping.
For instance, $\phi_p$ is advanced one time step $n$ to $n+1$ according to
\[\phi_{(n+1)}=\phi_{(n)}+\dt_{(n)}\left(\Mder\phi\right)_{(n)} +\half\dt_{(n)}^2\left(\Mdto\phi\right)_{(n)}+\sixth\frac{\dt_{(n)}^3}{\dt_{(n-1)}}\left(\Mdto\phi_{(n)}-{\Mdto\phi}_{(n-1)}\right).\]
The last, backward difference both increases the accuracy and stabilize the
scheme. The first temporal derivative of $\phi$ is obtained from the Bernoulli
equation. We then also obtain the local derivative, $\der\phi t$,
that is used to define a boundary value problem for the temporal derivatives
of the velocity. This is identical to the problem for the velocities 
themselves, given by \numm{Cauchyrel} or \numm{usrel} with $u$ and $v$
replaced by their local time derivatives. Formulas like the one above
are applied to the other principal unknowns $\xi$ and $\eta$.  
\item   Special treatment of corner points; invocation of analyticity.
\item Like most models of this kind some filtering is required in the nonlinear
      case to avoid growth of noise. A five point smoothing formula is applied
to this end.
\end{itemize}
In sum we have a ``moderately high order'' method that is lower order 
compared to the method \cite{Dold:1992}, but at the same time less 
restricted at the boundaries.
  
The computational cycle consists of the following main steps 
\begin{enumerate}
\item We know velocities (and more) at $t$. Time stepping by discrete 
      surface condition give $\phi$ (potential) and  particle positions 
      at the surface for $t+\Delta t$
\item $\phi$ at surface yields the tangential velocity at the surface
\item Crucial step: 
      Tangential velocity at surface and bottom condition (normal velocities)
      yield
      equations for the other velocity component through the integral 
      equation \numm{usrel} that is equivalent to the Laplace equation
\item $\partial\phi/\partial t$ is obtained from the Bernoulli equation \numm{Bernoulli}.
      The tangential component of the temporal derivative of the velocity is
      then obtained, in analogy to step 2, and
      the linear equation set from step 3 is solved with a new
      right hand side to obtain the remaining component of $\partial {\bf v} /\partial t$.
\item The Bernoulli equation is differentiated, materially, with respect to 
$t$ and   $\Mdto{\phi}$ is computed.
\item Now the first and second order Lagrangian derivatives of $\phi$, $\xi$
and $\eta$ are computed at $t+\dt$ and the cycle may repeat itself.
\end{enumerate}
The whole problem is then posed in terms of the position of the fluid boundary and the velocity potential there. Values of velocities within the fluid may
be obtained by choosing $z_p$ as
an interior point and put $\alpha=2\pi$ in the Cauchy relation 
\numm{Cauchyrel}, which then provide explicit expressions for $u_p$ and $v_p$.
In the linear case the procedure is substantially simplified since the 
geometry is constant and matrices involved in the third step may be 
computed and factorized only once.
 
\begin{figure}
\centerline{
\makebox{
\setlength{\unitlength}{1mm}
\begin{picture}(105,48)(0,-10)
\put(  5.25,  9.20){\epsfig{figure=BM.ps,width= 94.50\unitlength,height= 29.61\unitlength}}
\put( 52.50, 40.80){\makebox(20,10)[lb]{$z_p=x_p+\imunit y_p$}}
\put( 28.57, 21.04){$\nabla^2\phi=0$}
\put( 21.00, 41.80){$C$}
\end{picture}}
}
\caption[\ ]{ \label{Cauchydef}Definition sketch of computational domain in
BIM method
}
\end{figure}

\bibliographystyle{plain}  % typesettingsformatet

\bibliography{/hom/geirkp/tex/newref,/hom/geirkp/tex/runup}
\end{document}
